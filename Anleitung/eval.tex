\section{Auswertung}
\label{prot}
Allgemeiner Hinweis: Achten Sie bitte auf sinnvolle Bild- und Tabellenbeschriftungen und f\"ugen Sie, wenn notwendig, Quellenangaben hinzu. Wir empfehlen einf\"uhrende Kapitel (wie das Theorie- und Detektor-Kapitel) bereits vor Praktikumsbeginn zu schreiben.\\
\\
Ihr Protokoll sollte den folgenden Inhalt umfassen:
\begin{enumerate}
	\item Einleitung
	\item kurze Einf\"uhrung in das Standard-Modell und insbesondere in die Physik des Top-Quarks
	\item kurze Einf\"uhrung zum Large-Hadron-Collider sowie den Komponenten des CMS-Detektors (max. 3 Seiten)
	\item Auswertung der Aufgaben
	\begin{itemize}
		\item Charakterisierung von Top-Antitop-Ereignissen (inkl. Event-Displays)
		\item Auflistung und Erl\"auterung Ihrer Schnitte (inkl. der von Ihnen erstellen Graphiken, vor und nach den Schnitten)
		\item Ergebnis f\"ur $\sigma_{t\bar{t}}$, inklusive Fehlerrechnung und einem Vergleich mit einem offiziellen Ergebnis von CMS und/oder ATLAS
                \item Rekonstruktion des $t\overline{t}$-Systems und Bestimmung der Masse
                \item Diskussion des Versuches/Verbesserungen 
       \end{itemize}
	\item Zusammenfassung / Fazit
\end{enumerate}