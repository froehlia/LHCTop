\section{Datensatz und Monte-Carlo-Simulation}
\label{datasets}
Die Kollisionsrate der Protonen ist am LHC extrem hoch, sodass nicht alle Ereignisse gespeichert werden k\"onnen. Ein Trigger-System selektiert deshalb interessante Ereignisse und reduziert damit die totale Rate an Ereignissen. Der Trigger, der f\"ur diese Analyse benutzt wurde selektiert Ereignisse, die isolierte Myonen mit einem Transversalimpuls von $p_{T}>24$ GeV beinhalten. Das Sample mit echten Daten, die mit dem CMS-Detektor und dem Trigger aufgenommen wurden, hat 469384 Ereignisse. Dieses entspricht einer integrierten Luminosit\"at von 50 pb$^{-1}$. Weitere Details sind in Tabelle \ref{table_samples} zu finden.\\
Diese Tabelle beinhaltet dar\"uber hinaus Informationen \"uber die simulierten Samples. Die Anzahl der simulierten Ereignisse kann erheblich kleiner oder gr\"o\ss{}er sein als f\"ur 50 pb$^{-1}$ ben\"otigt, da die Ereignisse aus Simulationen entsprechend gewichtet werden k\"onnen. Dieses Gewicht ist als \textbf{EventWeight} - Variable in jedem simulierten Sample gespeichert (weitere Details dazu gibt es in Kapitel \ref{dataformat}). Damit die simulierten Samples jeweils einer integrierten Luminosit\"at von 50 pb$^{-1}$ entsprechen, m\"ussen alle Ereignisse mit ihrem \textbf{EventWeight} multipliziert werden.
\begin{small}
\begin{table}[h!]
  \centering
  \begin{tabular}{|l|c|c|c|}
    \hline 
Datei-Name & Prozess & $\#$Ereignisse & WQS \\ \hline\hline
data.root & Daten & 469384 & \\ \hline
ttbar.root & sim. $t\bar{t}$-Signal & 36941 & 165 pb  \\
wjets.root & sim. W+Jets Untergrund & 109737 & 31300 pb \\
dy.root & sim. Drell-Yan Untergrund & 77729 & 15800 pb \\
ww.root & sim. WW Untergrund & 4580 & 43 pb \\
wz.root & sim. WZ Untergrund & 3367 & 18 pb \\
zz.root & sim. ZZ Untergrund & 2421 & 6 pb \\
single\_top.root & sim. Single Top Untergrund & 5684 & 85 pb \\
qcd.root & sim. QCD Multijet Untegrund & 142 & $10^8$ pb \\
    \hline
  \end{tabular}
  \caption{Daten und simulierte MC-Samples.}
  \label{table_samples}
\end{table}
\end{small}
\\Wie Sie des Weiteren der Tabelle entnehmen k\"onnen, enth\"alt das QCD-Multijet-Sample nur noch sehr wenige getriggerte Ereignisse, trotzdem der Produktions-Wirkungsquerschnitt f\"ur diesen Prozess sehr gro\ss{} ist. Dieser Untergrund wird bereits aufgrund des Triggers stark reduziert, da dort ein isoliertes Myons verlangt wird.\\
Alle (ROOT-)Dateien sind in Unterordnern des HEPTutorial.tar.gz - Pakets zu finden.